%%%%%%%%%%%%%%%%%%%%%%%%%%%%%%%%%%%%%%%%%%%%%%%%%%%%%%%%%%%%%%%%%%%%%%%%%%%%%%%

\documentclass[12pt, a4paper, oneside]{article}

\input{/home/giatro/.config/user/giatro_packages.tex}

\input{/home/giatro/.config/user/giatro_macros.tex}

\title{Relatório do EP3 de Laboratório de Métodos Numéricos}
\date{\today}
\author{Lucas Paiolla Forastiere - 11221911}

%%%%%%%%%%%%%%%%%%%%%%%%%%%%%%%%%%%%%%%%%%%%%%%%%%%%%%%%%%%%%%%%%%%%%%%%%%%%%%%
%%%%%%%%%%%%%%%%%%%%%%%%%%%%%%%%%%%%%%%%%%%%%%%%%%%%%%%%%%%%%%%%%%%%%%%%%%%%%%%

\begin{document}

\maketitle

\section*{Parte 1}

A parte 1 foi bastante simples. Temos inicialmente o método \textsc{newton} 
que utiliza o Método das Diferenças Divididas de Newton para achar os coeficientes
de um polinômio interpolador. A função recebe um conjunto de \textsc{n} pontos
com suas coordenadas \textsc{x} e \textsc{y} e retorna um vetor de \textsc{n}
coeficientes do polinômio interpolador dos pontos - o primeiro elemento do vetor 
é o coeficiente livre e o último é o coeficiente de $x^n$.

Depois, temos uma função \textsc{f}, que retorna o valor do polinômio interpolador
para o ponto $x$. Ela recebe esse valor $x$ e também precisa de um vetor 
\textsc{x0} com os valores em $x$ que interpolamos. Por fim, ela precisa também 
dos coeficientes que calculamos por Newton. Passamos eles como um parâmetro pois
não queremos ter de calcular esses coeficientes em cada avaliação.

Após termos essas funções definidas, criamos as funções \textsc{TrapezoidRule} e\nl
\textsc{SimpsonRule} que são a implementação das fórmulas do Trapézio e de 
Simpson. Essas funções recebem um valor \textsc{n} indicando o número de intervalos,
valores \textsc{a} e \textsc{b} que são os limites da integração, os valores 
de $x$ dos pontos de interpolação, os coeficientes do polinômio interpolador e
o número de pontos interpolados.

Por fim, temos a \textsc{main} com o conjunto de pontos que se pede para interpolar
e é onde chamamos as outras funções para obter os resultados das interpolações
utilizando trapézio e Simpson.

\section*{Parte 2}

Essa parte foi também bastante simples de implementar, pois se criou apenas uma
função genérica com a implementação de Monte Carlo e depois cada questão foi 
implementada separadamente (falaremos disso mais tarde).

A função \textsc{montecarlo} recebe um valor \textsc{n}, indicando o número de pontos 
que vamos avaliar, recebe uma função \textsc{f} que é a função que estamos 
querendo integrar, recebe um vetor de pares ordenados chamado \textsc{limits} e 
recebe um valor \textsc{m} indicando a quantidade de dimensões da integral.

A função assume então as seguintes coisas: 
\begin{itemize}
\item A função \textsc{f} é uma função de 
\textsc{m} dimensões (ou seja, o parâmetro \textsc{*double} de \textsc{f} é um
vetor de \textsc{m} valores);
\item O vetor \textsc{limits} possui \textsc{m} pares ordenados em que 
\textsc{limits[0]} indica os limites inferior e superior da primeira integral,
\textsc{limits[1]} indica os limites inferior e superior da segunda integral
e assim por diante até \textsc{limits[m-1]}.
\end{itemize}

Utilizamos então o método de Monte Carlo e sorteamos cada coordenada do vetor
\textsc{X} que é o argumento da função \textsc{f}. Para fazer esse sorteio, 
utilizamos a transformação linear:
$$X=(\textsc{limits}_1-\textsc{limits}_0)\lambda+\textsc{limits}_0,$$

onde $\lambda$ é um vetor aleatório com cada coordenada 
entre $0$ e $1$, $\textsc{limits}_0$ é um vetor de coordenadas com os limites 
inferiores de cada uma das integrais em ordem e $\textsc{limits}_1$ a mesma coisa,
mas para os limites superiores. Todos esses vetores possuem \textsc{m} coordenadas.

Perceba que se $m=1$, então essa fórmula pode ser escrita como:
$$x=(b-a)\lambda+a$$

O que na prática essa transformação faz é pegar um ponto $X$ aleatório dentro 
do \textit{domínio de integração}. Nos utilizamos da hipótese que os números
aleatórios são gerados de forma uniforme e, portanto, temos que nosso método 
toma valores uniformemente dentro das possibilidades do domínio de integração
(que seria nosso espaço amostral).

Depois, a tarefa é simples, criamos, para cada integral pedida, uma função
que recebe um vetor \textsc{x} e utiliza esse vetor para calcular as funções
que queremos integrar em cada caso. Essas função são as funções
\textsc{f1}, \textsc{f2}, \textsc{f3} e \textsc{f4}. E depois, para cada 
uma delas, criamos uma função que vai criar um vetor de pares ordenados com 
os limites de cada integral associada ao problema e chamar o método 
de Monte Carlo para a sua respectiva função, com o determinado valor 
de \textsc{m} e os limites daquele problema. É mais fácil olhar diretamente no
código.

\subsection*{Adendo sobre a questão 3}

Todas as integrais possuem intervalos de integração finitos e relativamente 
pequenos quando comparamos com o número de pontos aleatórios que estamos 
avaliando. Entretanto, a terceira questão possui um intervalo de integração
infinito e daí simplesmente não conseguimos aplicar normalmente a transformação
linear desejada, pois o limite superior seria infinito (o que não conseguimos
representar no computador). Assim sendo, precisamos decidir um valor para o limite
superior que não comprometa nossa integração.

(Poderíamos também utilizar uma outra transformação para essa questão em particular,
mas devido à forma como construímos a função \textsc{montecarlo}, isso não seria
ideal.)

Portanto, para verificar qual seria o melhor limite superior, utilizou-se um 
método empírico (já que não sei fazer isso de forma analítica). Testou-se 
os seguintes valores de $b$ (o limite superior da integral):
\begin{itemize}
\item $1$;
\item $10$;
\item $100$;
\item $1000$;
\item $100000$.
\end{itemize}

Para cada um, fizemos cinco testes (para minimar as aleatoriedades do método de
Monte Carlo). Os resultados são monstrados a seguir:

\begin{center}
\begin{tabular}{|| c c c c c ||}
\hline
$1$ & $10$ & $100$ & $1000$ & $100000$ \\ [0.5ex]
\hline\hline
$0.632116$ & $1.000121$ & $0.999137$ & $1.000811$ & $0.989015$ \\ [0.1ex]
\hline
$0.632108$ & $1.000218$ & $0.999656$ & $1.003502$ & $1.030284$ \\ [0.1ex]
\hline
$0.632124$ & $0.999846$ & $1.000460$ & $1.000977$ & $0.996152$ \\ [0.1ex]
\hline
$0.632124$ & $1.000094$ & $1.000749$ & $1.001545$ & $1.012699$ \\ [0.1ex]
\hline
$0.632109$ & $0.999912$ & $0.999740$ & $0.997264$ & $1.006466$ \\ [1ex]
\hline
\end{tabular}
\end{center}

A função que fez esses testes é a \textsc{questao3testes}. Com base nisso,
optou-se por deixar o valor de $b=10$, pois vemos que com esse limite estamos
bem próximos do valor real da integral e também vemos que para valores de 
$b$ muito altos ou muito pequenos, começamos a ter imprecisões.

%%%%%%%%%%%%%%%%%%%%%%%%%%%%%%%%%%%%%%%%%%%%%%%%%%%%%%%%%%%%%%%%%%%%%%%%%%%%%%%

\end{document}
